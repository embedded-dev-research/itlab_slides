\documentclass{beamer}
\usetheme{Madrid}
\usepackage{graphicx}
\usepackage{amsmath}
\usepackage{hyperref}

\title[CV to OpenVINO]{From Images to OpenVINO Inference}
\author{Nesterov Alexander, Obolenskiy Arseniy}
\institute{ITLab}
\date{November 2025}

\setbeamertemplate{footline}{
  \leavevmode
  \hbox{%
    \begin{beamercolorbox}[wd=.45\paperwidth,ht=2.5ex,dp=1ex,leftskip=1em,center]{author in head/foot}
      \usebeamerfont{author in head/foot}\insertshortinstitute
    \end{beamercolorbox}%
    \begin{beamercolorbox}[wd=.45\paperwidth,ht=2.5ex,dp=1ex,leftskip=1em,center]{author in head/foot}
      \usebeamerfont{author in head/foot}\insertshorttitle
    \end{beamercolorbox}%
    \begin{beamercolorbox}[wd=.10\paperwidth,ht=2.5ex,dp=1ex,rightskip=1em,center]{author in head/foot}
      \usebeamerfont{author in head/foot}\insertframenumber{} / \inserttotalframenumber
    \end{beamercolorbox}}
  \vskip0pt
}

\AtBeginSection[]{
  \begin{frame}
    \centering
    \Huge\insertsection
  \end{frame}
}

\begin{document}

\begin{frame}
  \titlepage
\end{frame}

\begin{frame}{Contents}
  \small
  \tableofcontents[hideallsubsections]
\end{frame}

\section{Roadmap}

\begin{frame}{Goal for Today}
  \begin{itemize}
    \item Follow the path: \textbf{image $\rightarrow$ tensor $\rightarrow$ model artifact $\rightarrow$ ONNX $\rightarrow$ OpenVINO IR}.
    \item Keep only the math needed to reason about layout, shape, and precision.
    \item Learn how to check each hand-off before sending the model to OpenVINO.
  \end{itemize}
\end{frame}

\begin{frame}{Guiding Questions}
  \begin{itemize}
    \item What are the exact inputs and outputs?
    \item Where do layout or dtype changes happen, and do we log them?
    \item How do we prove ONNX/OpenVINO outputs match PyTorch?
    \item Which steps dominate latency, memory, or accuracy?
  \end{itemize}
\end{frame}

\begin{frame}{If This Is New...}
  \begin{itemize}
    \item You only need basic linear algebra (vectors, dot products) and the idea of a function.
    \item A \textbf{tensor} is just a multi-dimensional array; we care about its size and data type.
    \item Layout words (NCHW, NHWC) simply describe the axis order.
    \item Keep these mental anchors and the rest of the lecture will feel less abstract.
  \end{itemize}
\end{frame}

\section{Artifact and Interfaces}

\begin{frame}{Model = Portable Artifact}
  \begin{columns}
    \begin{column}{0.55\textwidth}
      \begin{itemize}
        \item Deliverable = graph + weights + fixed input/output description.
        \item Pre/post-processing belong to the contract, not \"misc\" code.
        \item Same bundle, same answer: store versions and short notes with it.
        \item Training notebooks can change; exported artifact must stay reproducible.
      \end{itemize}
    \end{column}
    \begin{column}{0.45\textwidth}
      \includegraphics[width=\textwidth]{images/dnn.png}
    \end{column}
  \end{columns}
\end{frame}

\begin{frame}{Anatomy of the Artifact}
  \begin{itemize}
    \item \textbf{Graph}: list of layers (operations) and how tensors flow between them.
    \item \textbf{Weights}: numbers learned during training; stored as tensors too.
    \item \textbf{I/O signature}: tensor shapes, layouts, dtypes, and names.
    \item \textbf{Metadata}: version, training recipe hash, preprocessing notes.
    \item Bundle them once and reuse the same bundle everywhere.
  \end{itemize}
\end{frame}

\begin{frame}{Inputs and Outputs}
  \begin{itemize}
    \item Input tensor: $\mathrm{images} \in \mathbb{R}^{N\times3\times H\times W}$, layout NCHW, dtype FP32.
    \item Normalize per channel: $x' = (x/255 - \mu)/\sigma$, $\mu=[0.485,0.456,0.406]$, $\sigma=[0.229,0.224,0.225]$.
    \item Example: single RGB photo $224\times224$ becomes tensor $[1,3,224,224]$ after preprocessing.
    \item Output tensor: $\mathrm{logits} \in \mathbb{R}^{N\times C}$ for softmax/argmax.
    \item Changing layout, dtype, or scale = effectively a different model.
  \end{itemize}
\end{frame}

\begin{frame}{Tensors and Memory}
  \begin{itemize}
    \item Memory footprint = elements $\times$ bytes per element.
    \item Example: $[1,3,224,224]$ FP32 $=1\cdot3\cdot224\cdot224\cdot4\approx0.6$ MB.
    \item FP32 (32-bit float) uses 4 bytes; FP16/BF16 use 2 bytes; INT8 uses 1 byte.
    \item FP16 halves activations and weights; INT8 shrinks by $\times4$ vs FP32.
    \item Keep mental math handy to size batches for GPUs, VPUs, and CPUs.
  \end{itemize}
\end{frame}

\begin{frame}{Shape Control}
  \begin{columns}
    \begin{column}{0.55\textwidth}
      \begin{itemize}
        \item Typical block: Conv $\rightarrow$ Activation $\rightarrow$ Pool $\rightarrow$ Linear.
        \item Conv2D height:
        \begin{equation*}
          H_{out}=\left\lfloor\frac{H_{in}+2p-d(k-1)-1}{s}\right\rfloor+1
        \end{equation*}
        \item $p$ = padding, $k$ = kernel, $s$ = stride, $d$ = dilation.
        \item Example: $H_{in}=224$, $k=3$, $p=1$, $s=2$, $d=1$ $\Rightarrow H_{out}=112$.
        \item Shape logs prevent silent layout swaps when exporting.
      \end{itemize}
    \end{column}
    \begin{column}{0.45\textwidth}
      \includegraphics[width=\textwidth]{images/lenet.png}
    \end{column}
  \end{columns}
\end{frame}

\begin{frame}{Where FLOPs Hide}
  \begin{itemize}
    \item Conv MACs (multiply-accumulates): $H_{out} W_{out} C_{out} (C_{in} k_H k_W / \text{groups})$.
    \item Fully connected (GEMM): $O(MNK)$; later layers dominate channels, early layers dominate spatial size.
    \item To save work: shrink spatial size early (stride/pool), use smaller kernels, or fewer channels.
    \item Estimate FLOPs before choosing hardware or batching strategy.
  \end{itemize}
\end{frame}

\section{Data Pipeline}

\begin{frame}{Data Pipeline}
  \framesubtitle{Preprocessing}
  \begin{columns}
    \begin{column}{0.5\textwidth}
      \begin{enumerate}
        \item Resize keeping aspect, then center-crop $224\times224$.
        \item Convert to RGB, scale to $[0,1]$, normalize per channel.
        \item Reorder HWC $\rightarrow$ CHW, append batch dim.
        \item Log resulting shape/range for every batch.
      \end{enumerate}
    \end{column}
    \begin{column}{0.5\textwidth}
      \includegraphics[width=\textwidth]{images/inference.png}
    \end{column}
  \end{columns}
\end{frame}

\begin{frame}{Data Pipeline}
  \framesubtitle{Postprocessing}
  \begin{itemize}
    \item Apply softmax with max-shift for numerical stability.
    \item Report top-$k$ classes plus confidence for audits.
    \item Keep `labels.txt` in sync with training order; mismatched indices produce wrong answers.
    \item Optional: threshold logits before argmax for "unknown" class.
  \end{itemize}
\end{frame}

\begin{frame}{One Pipeline for All Runtimes}
  \begin{itemize}
    \item Share the same `preprocess()`/`postprocess()` code in training, ONNX tests, and OpenVINO demos.
    \item Lock color space (RGB/BGR), resize mode, crop policy, and mean/std arrays in one config.
    \item Emit logs with tensor shapes and min/max values at every boundary.
    \item Most production bugs come from mismatched preprocessing rather than from the network graph.
  \end{itemize}
\end{frame}

\section{Export and Inspection}

\begin{frame}{ONNX in the Toolchain}
  \begin{columns}
    \begin{column}{0.55\textwidth}
      \begin{itemize}
        \item Open Neural Network Exchange = neutral graph + opset version.
        \item Lets PyTorch, TensorFlow, and JAX share the same artifact.
        \item Opset = versioned operator list; runtimes only implement certain versions.
        \item Inspect graphs with Netron or `onnxsim` before conversion.
      \end{itemize}
    \end{column}
    \begin{column}{0.45\textwidth}
      \includegraphics[width=0.9\textwidth]{images/frameworks.png}
    \end{column}
  \end{columns}
\end{frame}

\begin{frame}{ONNX to IR}
  \begin{columns}
    \begin{column}{0.6\textwidth}
      \begin{itemize}
        \item Model Optimizer converts ONNX to XML + BIN Intermediate Representation.
        \item XML stores the graph, BIN stores the weights—keep them together.
        \item Conversion folds constants, rewrites layouts, and attaches device hints.
        \item Keep opset/precision matrix handy before targeting edge hardware.
      \end{itemize}
    \end{column}
    \begin{column}{0.4\textwidth}
      \includegraphics[width=0.9\textwidth]{images/small_ir.png}
    \end{column}
  \end{columns}
\end{frame}

\begin{frame}{Visual Debugging}
  \begin{columns}
    \begin{column}{0.55\textwidth}
      \begin{itemize}
        \item Netron (free desktop/web viewer) shows tensor dims, kernel sizes, and data types in seconds.
        \item Use screenshots for before/after layout fixes.
        \item Pair with tensor dumps from PyTorch/ONNX/OpenVINO to locate diffs.
      \end{itemize}
    \end{column}
    \begin{column}{0.45\textwidth}
      \includegraphics[width=\textwidth]{images/openvino.png}
    \end{column}
  \end{columns}
\end{frame}

\begin{frame}[fragile]{PyTorch \texorpdfstring{$\rightarrow$}{->} ONNX}
\begin{verbatim}
import torch

model.eval()
dummy = torch.randn(1, 3, 224, 224)
torch.onnx.export(
  model, dummy, "model.onnx", opset_version=17,
  input_names=["images"], output_names=["logits"],
  dynamic_axes={"images": {0: "N"}, "logits": {0:  "N"}}
)
\end{verbatim}
  \begin{itemize}
    \item `eval()` locks BatchNorm/Dropout behavior.
    \item Dummy tensor must reflect real preprocessing and dtype.
    \item Dynamic axes keep batch size flexible without retraining.
  \end{itemize}
\end{frame}

\begin{frame}{Validate the Export}
  \begin{itemize}
    \item Compare PyTorch vs ONNX Runtime on 10--20 real images.
    \item Metrics: top-1 match, cosine similarity (should be near 1), and L2 distance.
    \item Set tolerance $|y_{ref} - y_{onnx}| \le 10^{-4}$ (FP32) or tighter for FP16.
    \item Fix seeds and record library versions to make regressions traceable.
  \end{itemize}
\end{frame}

\section{Performance and Reliability}

\begin{frame}{Latency and Throughput}
  \begin{itemize}
    \item Warm up 5--10 iterations before timing.
    \item Latency per sample: $\bar{t} = \sum \Delta t_i / K$ (average over $K$ runs).
    \item Throughput: $\mathrm{fps} = (N\cdot K)/\sum \Delta t_i$ ($N$ = batch size).
    \item Report P50/P95, mention batch size, precision, threading flags, and device.
    \item Batch $=1$ for real-time latency; batch $>1$ for throughput demos.
  \end{itemize}
\end{frame}

\begin{frame}{Precision and Quantization}
  \begin{itemize}
    \item FP32: baseline accuracy, highest compute and memory.
    \item FP16/BF16: half memory, often lossless on modern accelerators.
    \item INT8 PTQ: $q=\operatorname{round}(x/\text{scale}+z_p)$, recover via $x \approx \text{scale}(q-z_p)$.
    \item Mixed precision keeps sensitive ops (first/last layers, softmax) in FP32.
  \end{itemize}
\end{frame}

\begin{frame}{Calibration Set}
  \begin{itemize}
    \item Representative set = small but realistic snapshot of production data.
    \item Collect 100--200 domain images covering lighting, poses, object sizes.
    \item Reuse for INT8 calibration and regression checks after every optimization.
    \item Track histograms/min-max per channel to detect drift.
    \item Poor sampling leads to quantization drops that benchmarks might miss.
  \end{itemize}
\end{frame}

\begin{frame}{Typical Failures}
  \begin{itemize}
    \item NHWC vs NCHW swaps, wrong normalization constants, or missing batch dimension.
    \item Resize mismatch: center crop vs letterbox vs stretch.
    \item Misaligned `labels.txt` or off-by-one IDs (class names shift).
    \item Debug with Netron, per-layer tensor diffs, and logging of min/max values.
  \end{itemize}
\end{frame}

\begin{frame}{Key Messages}
  \begin{itemize}
    \item Treat the model artifact and I/O contract as a single deliverable.
    \item Validate every hop: PyTorch $\rightarrow$ ONNX $\rightarrow$ OpenVINO IR.
    \item Benchmark under realistic loads and document precision choices.
    \item Representative data and logging catch regressions before deployment.
  \end{itemize}
\end{frame}

\end{document}
