\documentclass{beamer}

% Theme choice
\usetheme{Madrid}

% Optional packages
\usepackage{graphicx} % For including images
\usepackage{amsmath}  % For math symbols and formulas
\usepackage{hyperref} % For hyperlinks
\usepackage{listings} % For code listings
\usepackage{xcolor}   % For color definitions

\lstdefinestyle{CStyle}{
  language=C,                    % Set the language to C
  basicstyle=\ttfamily\footnotesize\linespread{0.9}\tiny, % Set font style and size
  keywordstyle=\color{blue},      % Color of keywords
  commentstyle=\color{gray},      % Color of comments
  stringstyle=\color{red},        % Color of strings
  showstringspaces=false,         % Do not mark spaces in strings
  breaklines=true,                % Enable line breaks at appropriate places
  breakatwhitespace=false,        % Break lines at any character, not just whitespace
  numbers=left,                   % Show line numbers on the left
  numberstyle=\tiny\color{gray},  % Style for line numbers
  tabsize=4,                      % Set tab width
  keepspaces=true,                % Keep indentation spaces
  frame=single,                   % Add a border around the code
  aboveskip=0pt,                  % Reduce space above the code block
  belowskip=0pt,                   % Reduce space below the code block
  xleftmargin=7.5pt,                      % Add left padding (approx. 2.8mm or 10px)
  xrightmargin=15pt,                      % Add left padding (approx. 2.8mm or 10px)
}

\title[CMake]{CMake}
\author{Obolenskiy Arseniy, Nesterov Alexander}
\institute{ITLab}

\date{\today}

% Redefine the footline to display both the short title and the org name
\setbeamertemplate{footline}{
  \leavevmode%
  \hbox{%
    \begin{beamercolorbox}[wd=.45\paperwidth,ht=2.5ex,dp=1ex,leftskip=1em,center]{author in head/foot}%
        \usebeamerfont{author in head/foot}\insertshortinstitute% Displays the university name
    \end{beamercolorbox}%
    \begin{beamercolorbox}[wd=.45\paperwidth,ht=2.5ex,dp=1ex,leftskip=1em,center]{author in head/foot}%
      \usebeamerfont{author in head/foot}\insertshorttitle% Displays the short title
    \end{beamercolorbox}%
    \begin{beamercolorbox}[wd=.1\paperwidth,ht=2.5ex,dp=1ex,rightskip=1em,center]{author in head/foot}%
      \usebeamerfont{author in head/foot}\insertframenumber{} / \inserttotalframenumber%
    \end{beamercolorbox}}%
  \vskip0pt%
}

\begin{document}

\begin{frame}
    \titlepage%
\end{frame}

\begin{frame}{Contents}
    \tableofcontents
\end{frame}

\section{Building C++ projects}

\begin{frame}[fragile]{C++ ``Hello, World'' example}
  \lstset{style=CStyle, caption=Hello World example}
  \begin{lstlisting}
    #include <iostream>

    int main() {
        std::cout << "Hello, World!" << std::endl;
        return 0;
    }
  \end{lstlisting}
\end{frame}

\begin{frame}[fragile]{Building simple main.cpp on UNIX}
  \begin{itemize}
    \item Open a terminal.
    \item Navigate to the directory containing \texttt{main.cpp}.
    \item Compile the program using \texttt{g++}:
    \begin{lstlisting}[language=bash]
g++ -o hello main.cpp
    \end{lstlisting}
    \item Run the executable:
    \begin{lstlisting}[language=bash]
./hello
    \end{lstlisting}
  \end{itemize}
\end{frame}

\begin{frame}[fragile]{Building simple main.cpp on Windows}
  \begin{itemize}
    \item Open the Command Prompt.
    \item Navigate to the directory containing \texttt{main.cpp}.
    \item If using MinGW\@:
    \begin{lstlisting}[language=bash]
g++ -o hello.exe main.cpp
hello.exe
    \end{lstlisting}
    \item If using Visual Studio Developer Command Prompt:
    \begin{lstlisting}[language=bash]
cl /EHsc main.cpp
main.exe
    \end{lstlisting}
  \end{itemize}
\end{frame}

\begin{frame}[fragile]{Simple library example}
  Assuming there is a simple library that contains a function \texttt{add}:

  \lstset{style=CStyle, caption=add.h}
  \begin{lstlisting}
#ifndef ADD_H
#define ADD_H

int add(int x, int y);

#endif // ADD_H
  \end{lstlisting}

  \lstset{style=CStyle, caption=add.cpp}
  \begin{lstlisting}
  #include "add.h"
  
  int add(int x, int y) {
      return x + y;
  }
  \end{lstlisting}

  \lstset{style=CStyle, caption=main.cpp}
  \begin{lstlisting}
    #include <iostream>
    #include "add.h"
    
    int main() {
        int result = add(5, 3);
        std::cout << "5 + 3 = " << result << std::endl;
        return 0;
    }
  \end{lstlisting}
\end{frame}

\begin{frame}[fragile]{Building simple main.cpp with add library on UNIX}
  \begin{itemize}
    \item Compile \texttt{add.cpp} into an object file:
    \begin{lstlisting}[language=bash]
g++ -c add.cpp
    \end{lstlisting}
    \item Compile \texttt{main.cpp} into an object file:
    \begin{lstlisting}[language=bash]
g++ -c main.cpp
    \end{lstlisting}
    \item Link the object files into an executable:
    \begin{lstlisting}[language=bash]
g++ -o program main.o add.o
    \end{lstlisting}
    \item Run the executable:
    \begin{lstlisting}[language=bash]
./program
    \end{lstlisting}
  \end{itemize}
\end{frame}

\begin{frame}[fragile]{Building simple main.cpp with add library on Windows}
  Using MinGW\@:
  \begin{itemize}
    \item Compile \texttt{add.cpp}:
    \begin{lstlisting}[language=bash]
g++ -c add.cpp
    \end{lstlisting}
    \item Compile \texttt{main.cpp}:
    \begin{lstlisting}[language=bash]
g++ -c main.cpp
    \end{lstlisting}
    \item Link the object files:
    \begin{lstlisting}[language=bash]
g++ -o program.exe main.o add.o
    \end{lstlisting}
  \end{itemize}

  Using Visual Studio Developer Command Prompt:
  \begin{itemize}
    \item Compile and link:
    \begin{lstlisting}[language=bash]
cl /EHsc main.cpp add.cpp
    \end{lstlisting}
    \item Run the executable:
    \begin{lstlisting}[language=bash]
program.exe
    \end{lstlisting}
  \end{itemize}
\end{frame}

\section{Build systems history}

\begin{frame}{Build systems}
  There are various build systems for different environments:

  \begin{itemize}
    \item Make
    \item Autoconf/Automake
    \item CMake
    \item Ninja
    \item Meson
    \item Bazel
    \item Visual Studio (MSBuild)
  \end{itemize}
\end{frame}

\begin{frame}{Make}
  \begin{itemize}
    \item Traditional build tool using Makefiles.
    \item Defines rules and dependencies for compiling code.
    \item Widely used on UNIX systems.
    \item Simple but can become complex for large projects.
  \end{itemize}
\end{frame}

\begin{frame}{Autoconf/Automake}
  Autoconf is the colution to automatically generate Makefiles.
  \begin{itemize}
    \item Part of the GNU build system.
    \item Generates portable Makefiles.
    \item Handles platform-specific configurations.
    \item Useful for open-source projects targeting multiple UNIX-like systems.
  \end{itemize}
\end{frame}

\begin{frame}{CMake}
  \begin{itemize}
    \item Cross-platform build system generator.
    \item Generates native build files (Makefiles, Visual Studio solutions, etc.).
    \item Supports complex project configurations.
    \item Widely adopted in both open-source and commercial projects.
  \end{itemize}
\end{frame}

\begin{frame}{Ninja}
  \begin{itemize}
    \item Focused on speed and efficiency.
    \item Uses a simple build file format.
    \item Often used as a backend by higher-level build systems like CMake and Meson.
    \item Not intended to be written by hand.
  \end{itemize}
\end{frame}

\begin{frame}{Meson}
  \begin{itemize}
    \item High-level build system with a simple syntax.
    \item Uses Ninja as its default backend.
    \item Designed for fast and user-friendly builds.
    \item Supports multiple programming languages.
  \end{itemize}
\end{frame}

\begin{frame}{Bazel}
  \begin{itemize}
    \item Developed by Google for large-scale projects.
    \item Focuses on build correctness and reproducibility.
    \item Supports multiple languages and platforms.
    \item Uses a domain-specific language for build definitions.
  \end{itemize}
\end{frame}

\begin{frame}{Visual Studio/MSBuild}
  \begin{itemize}
    \item Microsoft native build system and toolchain on Windows.
    \item Uses MSBuild with \texttt{.sln}/\texttt{.vcxproj} project files.
    \item Deep IDE integration; also builds from CLI via \texttt{msbuild} or \texttt{cmake --build}.
    \item Often generated by CMake for Windows targets.
  \end{itemize}
\end{frame}

\section{CMake}

\begin{frame}{Why CMake?}
  \begin{itemize}
    \item Cross-platform compatibility.
    \item Generates native build systems.
    \item Handles complex build requirements.
    \item Strong community support and documentation.
    \item Integrates well with IDEs and other tools.
  \end{itemize}
\end{frame}

\begin{frame}[fragile]{CMake with Visual Studio generator}
  Generate a Visual Studio solution and build from the command line.
  \begin{itemize}
    \item Configure (choose VS version and architecture):
    \begin{lstlisting}[language=bash]
cmake -S . -B build -G "Visual Studio 18 2026" -A x64
    \end{lstlisting}
    \item Build (Debug/Release via --config):
    \begin{lstlisting}[language=bash]
cmake --build build --config Release
    \end{lstlisting}
    \item Run the executable:
    \begin{lstlisting}[language=bash]
build\\Release\\main.exe
    \end{lstlisting}
    \item Alternatively open \texttt{build/HelloAdd.sln} in Visual Studio and build F7.
  \end{itemize}
\end{frame}

\begin{frame}[fragile]{Download CMake}
  \begin{itemize}
    \item Official website: \url{https://cmake.org}
    \item Available for Windows, macOS, and Linux.
    \item Installation via package managers:
    \begin{itemize}
      \item On Ubuntu:
      \begin{lstlisting}[language=bash]
sudo apt-get install cmake
      \end{lstlisting}
      \item On macOS with Homebrew:
      \begin{lstlisting}[language=bash]
brew install cmake
      \end{lstlisting}
    \end{itemize}
  \end{itemize}
\end{frame}

\begin{frame}[fragile]{CMake script for main.cpp with add library}
  \begin{block}{CMakeLists.txt}
    \begin{lstlisting}
cmake_minimum_required(VERSION 3.0)
project(HelloAdd)

add_library(add add.cpp)
add_executable(main main.cpp)
target_link_libraries(main add)
    \end{lstlisting}
  \end{block}
\end{frame}

\begin{frame}[fragile]{CMake configure and CMake build commands}
  \begin{itemize}
    \item Create a build directory:
    \begin{lstlisting}[language=bash]
mkdir build
cd build
    \end{lstlisting}
    \item Configure the project:
    \begin{lstlisting}[language=bash]
cmake ..
    \end{lstlisting}
    \item Build the project:
    \begin{lstlisting}[language=bash]
cmake --build .
    \end{lstlisting}
    \item Run the executable:
    \begin{lstlisting}[language=bash]
./main    # On UNIX
main.exe  # On Windows
    \end{lstlisting}
  \end{itemize}
\end{frame}

\begin{frame}[fragile]{CMake configure and CMake build commands}
  Usage of \texttt{-S} and \texttt{-B} is encouraged:

  \begin{itemize}
    \item Configure the project:
    \begin{lstlisting}[language=bash]
cmake -S . -B build
    \end{lstlisting}
    \item Build the project:
    \begin{lstlisting}[language=bash]
cmake --build build
    \end{lstlisting}
    \item Run the executable:
    \begin{lstlisting}[language=bash]
./build/main    # On UNIX
build/main.exe  # On Windows
    \end{lstlisting}
  \end{itemize}
\end{frame}

\begin{frame}{CMake configure}
  \begin{itemize}
    \item Confgure
    \item Build
  \end{itemize}
\end{frame}

\begin{frame}{CMake configure}
  In a typical CMake-based build process, there are two primary stages: the configure step and the build step.
  \begin{itemize}
    \item Purpose: To inspect the system, check dependencies, and generate platform-specific build files (like Makefile, ninja files, or Visual Studio project files).
    \item Outcome: After this step, the necessary files for building the project are ready and platform-specific. The project isn't compiled yet, but CMake has configured everything based on the system environment, the user options, and the project setup.
  \end{itemize}
\end{frame}

\begin{frame}{CMake configure (continued)}
  \footnotesize
  What Happens on CMake confguration stage?
  \begin{itemize}
    \item CMakeLists.txt Processing: CMake reads the CMakeLists.txt files in your project. This file describes how the project should be built, which dependencies are needed, which source files to compile, etc.
    \item Checking Dependencies: CMake will check for external libraries and dependencies, ensuring they are installed and can be found (e.g., checking for required packages, libraries).
    \item Compiler and Toolchain Discovery: CMake will determine which compilers (e.g., GCC, Clang, MSVC) and toolchains to use based on the environment or user input.
    \item Configuration Options: During this step, you can also pass configuration options to CMake using the -D flag, which can influence how the project is built (e.g., enabling/disabling certain features or setting paths).
    \item Build File Generation: CMake generates the actual build system files (like Makefile, Ninja files, or Visual Studio solution files) in the output directory (typically in a build/ folder). These files are specific to the build system chosen by the user (e.g., Make, Ninja, MSBuild).
  \end{itemize}
\end{frame}

\begin{frame}{CMake build}
  The build step is where the actual compilation and linking take place.
  \begin{itemize}
    \item Purpose: To compile the source code into binary executables, libraries, or other artifacts based on the configuration done in the previous step.
    \item Outcome: After this step, the output files (executables, libraries, etc.) are created and are ready to be run or installed.
  \end{itemize}
\end{frame}

\begin{frame}{CMake build (continued)}
  What Happens on CMake build stage?
  \begin{itemize}
    \item Calling Build System: The generated build files from the configure step (like Makefile or ninja.build files) are invoked by CMake to compile the code.
    \item Compiling Source Code: The build system uses the appropriate compiler to compile the source code into object files.
    \item Linking: Once the object files are generated, the linker combines them into final binaries (e.g., an executable or shared library).
    \item Error/Warning Reporting: If there are any syntax errors, missing files, or other issues, they will typically surface during this step, as the code is actively being compiled.
    \item Rebuilding: CMake build system will track changes to source files. If you make changes to specific files and run the build step again, only the modified files are recompiled (i.e., incremental builds).
  \end{itemize}
\end{frame}

\begin{frame}[fragile]{Modern target-based CMake}
  Prefer target properties over global variables.
  \begin{block}{CMakeLists.txt}
    \begin{lstlisting}
cmake_minimum_required(VERSION 3.16)
project(App)

add_library(lib STATIC src/lib.cpp)
target_include_directories(lib PUBLIC include)

add_executable(app src/main.cpp)
target_link_libraries(app PRIVATE lib)
target_compile_features(app PRIVATE cxx_std_20)
    \end{lstlisting}
  \end{block}
  PRIVATE/PUBLIC/INTERFACE control propagation of usage requirements.
\end{frame}

\begin{frame}[fragile]{Project layout and add\_subdirectory}
  Example tree and minimal setup.
  \begin{lstlisting}
project/
  CMakeLists.txt
  src/ CMakeLists.txt main.cpp
  lib/ CMakeLists.txt include/lib/lib.h lib.cpp
  \end{lstlisting}
  \begin{block}{Top-level CMakeLists.txt}
    \begin{lstlisting}
cmake_minimum_required(VERSION 3.16)
project(App)
add_subdirectory(lib)
add_subdirectory(src)
    \end{lstlisting}
  \end{block}
\end{frame}

\begin{frame}[fragile]{Dependencies: find\_package vs FetchContent}
  Use packages when available; fetch when needed.
  \begin{block}{find\_package (fmt)}
    \begin{lstlisting}
find_package(fmt CONFIG REQUIRED)
target_link_libraries(app PRIVATE fmt::fmt)
    \end{lstlisting}
  \end{block}
  \begin{block}{FetchContent (GoogleTest)}
    \begin{lstlisting}
include(FetchContent)
FetchContent_Declare(
  googletest
  URL https://github.com/google/googletest/archive/refs/tags/v1.14.0.zip)
FetchContent_MakeAvailable(googletest)
    \end{lstlisting}
  \end{block}
\end{frame}

\begin{frame}[fragile]{Testing with CTest and GoogleTest}
  Integrate tests and run via CTest.
  \begin{block}{CMakeLists.txt}
    \begin{lstlisting}
enable_testing()
add_executable(app_tests tests/tests.cpp)
target_link_libraries(app_tests PRIVATE gtest_main)
include(GoogleTest)
gtest_discover_tests(app_tests)
    \end{lstlisting}
  \end{block}
  Run tests:
  \begin{lstlisting}[language=bash]
ctest --output-on-failure -j
  \end{lstlisting}
\end{frame}

\begin{frame}[fragile]{CMake Presets (CMakePresets.json)}
  Shareable configure/build/test profiles for CLI and IDEs.
  \begin{block}{CMakePresets.json}
    \begin{lstlisting}
{
  "version": 3,
  "cmakeMinimumRequired": {"major": 3, "minor": 22},
  "configurePresets": [
    {"name":"ninja-debug","generator":"Ninja",
     "binaryDir":"build/debug",
     "cacheVariables":{"CMAKE_BUILD_TYPE":"Debug"}}
  ],
  "buildPresets": [
    {"name":"build-debug","configurePreset":"ninja-debug"}
  ],
  "testPresets": [
    {"name":"test-debug","configurePreset":"ninja-debug"}
  ]
}
    \end{lstlisting}
  \end{block}
  \begin{lstlisting}[language=bash]
cmake --preset ninja-debug
cmake --build --preset build-debug
ctest --preset test-debug
  \end{lstlisting}
\end{frame}

\begin{frame}[fragile]{Cross-compiling and toolchains}
  Point CMake to a toolchain file for non-host targets.
  \begin{lstlisting}[language=bash]
cmake -S . -B build-arm \
  -DCMAKE_TOOLCHAIN_FILE=toolchains/arm-none-eabi.cmake
  \end{lstlisting}
  \begin{block}{toolchains/arm-none-eabi.cmake}
    \begin{lstlisting}
set(CMAKE_SYSTEM_NAME Generic)
set(CMAKE_TRY_COMPILE_TARGET_TYPE STATIC_LIBRARY)
set(CMAKE_C_COMPILER arm-none-eabi-gcc)
set(CMAKE_CXX_COMPILER arm-none-eabi-g++)
    \end{lstlisting}
  \end{block}
\end{frame}

\begin{frame}[fragile]{Install, export, and CPack}
  Install targets and generate simple packages.
  \begin{block}{CMakeLists.txt}
    \begin{lstlisting}
install(TARGETS app EXPORT appTargets RUNTIME DESTINATION bin)
install(EXPORT appTargets NAMESPACE app:: DESTINATION lib/cmake/app)
include(CPack)
    \end{lstlisting}
  \end{block}
  Create an archive package:
  \begin{lstlisting}[language=bash]
cpack -G ZIP
  \end{lstlisting}
\end{frame}

\begin{frame}[fragile]{Examples and demo}
  Example projects are included with this lecture under:
  \begin{itemize}
    \item \texttt{02-cmake/examples/steps} — step-by-step progression
    \item \texttt{02-cmake/examples/hello} — minimal project
    \item \texttt{02-cmake/examples/modern} — target-based project with tests and presets
  \end{itemize}
  Quick start (Unix, Ninja):
  \begin{lstlisting}[language=bash]
cd 02-cmake/examples/steps/02-hello-cmake
cmake -S . -B build -G Ninja -D CMAKE_BUILD_TYPE=Debug
cmake --build build
  \end{lstlisting}
  Quick start (Windows, Visual Studio):
  \begin{lstlisting}[language=bash]
cd 02-cmake\examples\steps\02-hello-cmake
cmake -S . -B build -G "Visual Studio 17 2022" -A x64
cmake --build build --config Debug
  \end{lstlisting}
\end{frame}

\begin{frame}
    \centering
    \Huge{Thank You!}
\end{frame}

\begin{frame}{References}
  \begin{enumerate}
    \item CMake Download \href{https://cmake.org/download/}{https://cmake.org/download/}
    \item CMake Tutorial \href{https://cmake.org/cmake/help/latest/guide/tutorial/index.html}{https://cmake.org/cmake/help/latest/guide/tutorial/index.html}
    \item An Introduction to Modern CMake \href{https://cliutils.gitlab.io/modern-cmake/README.html}{https://cliutils.gitlab.io/modern-cmake/README.html}
  \end{enumerate}
\end{frame}

\end{document}
