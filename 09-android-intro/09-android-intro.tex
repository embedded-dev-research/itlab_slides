\documentclass{beamer}

% Theme choice
\usetheme{Madrid}

% Optional packages
\usepackage[utf8]{inputenc}
\usepackage[T1]{fontenc}
\usepackage[english]{babel}
\usepackage{graphicx} % For including images
\usepackage{amsmath}  % For math symbols and formulas
\usepackage{hyperref} % For hyperlinks
\usepackage{listings} % For code snippets

\lstset{
  basicstyle=\ttfamily\footnotesize,
  breaklines=true,
  columns=fullflexible,
  keepspaces=true
}

\title[OpenVINO on Android]{OpenVINO on Android: quick guide}
\author{Obolenskiy Arseniy, Nesterov Alexander}
\institute{ITLab}
\date{\today}

% Redefine the footline to display both the short title and the org name
\setbeamertemplate{footline}{
  \leavevmode%
  \hbox{%
    \begin{beamercolorbox}[wd=.45\paperwidth,ht=2.5ex,dp=1ex,leftskip=1em,center]{author in head/foot}%
        \usebeamerfont{author in head/foot}\insertshortinstitute%
    \end{beamercolorbox}%
    \begin{beamercolorbox}[wd=.45\paperwidth,ht=2.5ex,dp=1ex,leftskip=1em,center]{author in head/foot}%
      \usebeamerfont{author in head/foot}\insertshorttitle%
    \end{beamercolorbox}%
    \begin{beamercolorbox}[wd=.1\paperwidth,ht=2.5ex,dp=1ex,rightskip=1em,center]{author in head/foot}%
      \usebeamerfont{author in head/foot}\insertframenumber{} / \inserttotalframenumber%
    \end{beamercolorbox}}%
  \vskip0pt%
}

\AtBeginSection[]{
  \begin{frame}
    \centering
    \Huge\insertsection%
  \end{frame}
}

\begin{document}

\begin{frame}
  \titlepage%
\end{frame}

\section{Platform}
\begin{frame}{Android stack (bottom to top)}
  \begin{itemize}
    \item Linux kernel: drivers, memory, threading and security primitives.
    \item HAL layer gives stable APIs to diverse chips; keeps upper layers portable
    \item ART runtime executes app bytecode/Dex
    \item Native C/C++ libs: media/graphics; we hook via NDK for speed.
    \item Java/Kotlin API framework + services: usual app layer on top.
    \item Takeaway: we can reach native performance while staying compatible.
  \end{itemize}
\end{frame}

\section{Tools}
\begin{frame}{SDK vs NDK}
  \begin{itemize}
    \item SDK stack: Java/Kotlin UI + app logic; produces APKs
    \item NDK toolkit builds C/C++ \texttt{.so}; OpenVINO core is C++ so NDK is mandatory
    \item Workflow: cross-compile on host for target ABI (ARM/x86\_64) via NDK toolchain.
  \end{itemize}
\end{frame}

\begin{frame}{What to install}
  \begin{itemize}
    \item Android NDK (e.g., r28c) --- includes \texttt{android.toolchain.cmake}.
    \item Platform Tools (ADB) --- push binaries, run shell, grab logs.
    \item CMake + Ninja on host --- configure/build quickly; any modern CMake works.
    \item Optional: Android Studio only if you plan to package an APK later.
  \end{itemize}
\end{frame}

\begin{frame}{What is ADB}
  \begin{itemize}
    \item Android Debug Bridge = client/server/daemon trio for device control over USB/Wi‑Fi.
    \item Core uses: file transfer (\texttt{adb push/pull}), remote shell (\texttt{adb shell}), install/uninstall APKs, logcat, port forwarding.
    \item ADB must see the device: enable Developer options + USB debugging; check with \texttt{adb devices}.
    \item For our flow: we use it to copy OpenVINO libs/tools and run \texttt{benchmark\_app} on-device.
  \end{itemize}
\end{frame}

\begin{frame}[fragile]{No device? Use an emulator (CLI)}
  \begin{itemize}
    \item Install Android SDK command-line tools; add \texttt{sdkmanager}, \texttt{avdmanager}, \texttt{emulator} to \texttt{PATH}.
    \item Download image (example): \texttt{sdkmanager ``system-images;android-35;google\_apis;x86\_64''}.
    \item Create an AVD using \texttt{avdmanager create avd -n ov-emul -k ``system-images;android-35;google\_apis;x86\_64''}
    \item Run emulator headless: \texttt{emulator -avd ov-emul -no-window -gpu swiftshader\_indirect}.
    \item Verify ADB sees it: \texttt{adb devices} should list the virtual device; then use the same push/run steps as for hardware
  \end{itemize}
\end{frame}

\section{Hello World (native)}
\begin{frame}[fragile]{Build a tiny C++ binary}
  \begin{itemize}
    \item Minimal source:
  \end{itemize}
  \begin{lstlisting}
// hello.cpp
#include <iostream>
int main() { std::cout << "Hello Android!\n"; }
  \end{lstlisting}
  \vspace{0.5em}
  \begin{lstlisting}
cmake -B build -G Ninja \
  -DANDROID_ABI=$CURRENT_ANDROID_ABI \
  -DANDROID_PLATFORM=$CURRENT_ANDROID_PLATFORM \
  -DCMAKE_TOOLCHAIN_FILE=$CURRENT_CMAKE_TOOLCHAIN_FILE \
  -DCMAKE_BUILD_TYPE=Release
cmake --build build --parallel
  \end{lstlisting}
  \begin{itemize}
    \item Result: \texttt{build/hello} ELF for the target ABI.\@
  \end{itemize}
\end{frame}

\begin{frame}[fragile]{Run Hello World on device}
  \begin{itemize}
    \item Push and execute via ADB (commands below).\@
  \end{itemize}
  \begin{lstlisting}
adb push build/hello /data/local/tmp/
adb shell "chmod +x /data/local/tmp/hello && /data/local/tmp/hello"
  \end{lstlisting}
  \begin{itemize}
    \item Expect stdout: \texttt{Hello Android!}
    \item If it fails: check ABI/API level and \texttt{LD\_LIBRARY\_PATH} (for libc++ if dynamically linked).
  \end{itemize}
\end{frame}

\section{Homework}
\begin{frame}{HW — Hello Android}
  \begin{itemize}
    \item Prepare env: install NDK + Platform Tools, set \texttt{CURRENT\_ANDROID\_*}; emulator allowed if no device.
    \item Build \texttt{hello.cpp} via CMake/NDK (recipe in slides); get \texttt{build/hello} for your ABI/API level.
    \item Run on device/emulator: \texttt{adb push}, \texttt{chmod +x}, execute; expect \texttt{Hello Android!} on stdout.
    \item Submit: short note (ABI + API level, commands used, screenshot/log of output); if something failed, describe it separately.
    \item Bonus: static link (\texttt{ANDROID\_STL=c++\_static}) or wrap the binary into a minimal APK via Android Studio.
  \end{itemize}
\end{frame}

\section{Environment}
\begin{frame}{Key variables}
  \begin{itemize}
    \item \texttt{ANDROID\_NDK\_PATH}, \texttt{ANDROID\_TOOLS\_PATH}.
    \item \texttt{CURRENT\_ANDROID\_ABI} (typically \texttt{arm64-v8a}; must match device).
    \item \texttt{CURRENT\_ANDROID\_PLATFORM} (API level, e.g., 35; not higher than device).
    \item \texttt{CURRENT\_ANDROID\_STL} (\texttt{c++\_shared} for shared libc++).
    \item \texttt{CURRENT\_CMAKE\_TOOLCHAIN\_FILE} (\ldots/android.toolchain.cmake) to enable cross-build.
    \item Wrong values \textrightarrow{} link/runtime errors; double-check before configuring.
  \end{itemize}
\end{frame}

\begin{frame}{Supported ABIs}
  \begin{itemize}
    \item \texttt{arm64-v8a}, \texttt{armeabi-v7a}, \texttt{x86\_64}, \texttt{riscv64} (experimental).
    \item Detect on device: \texttt{adb shell getprop ro.product.cpu.abi}.
    \item Build only what you need (faster builds, smaller artifact set).
  \end{itemize}
\end{frame}

\section{Build}
\begin{frame}[fragile]{oneTBB}
  \begin{itemize}
    \item Clone the oneTBB repo (threading backend OpenVINO depends on).
    \item Configure with Android toolchain; disable tests to speed up.
    \item CMake example:
  \end{itemize}
  \begin{lstlisting}
cmake -B build -G Ninja \
  -DANDROID_ABI=arm64-v8a \
  -DANDROID_PLATFORM=35 \
  -DANDROID_STL=c++_shared \
  -DCMAKE_TOOLCHAIN_FILE=$CURRENT_CMAKE_TOOLCHAIN_FILE \
  -DCMAKE_BUILD_TYPE=Release \
  -DTBB_TEST=OFF
cmake --build build --parallel
cmake --install build --prefix $PWD/install
  \end{lstlisting}
\end{frame}

\begin{frame}[fragile]{OpenVINO}
  \begin{lstlisting}
cmake -B build -G Ninja \
  -DANDROID_ABI=$CURRENT_ANDROID_ABI \
  -DANDROID_PLATFORM=$CURRENT_ANDROID_PLATFORM \
  -DANDROID_STL=$CURRENT_ANDROID_STL \
  -DCMAKE_TOOLCHAIN_FILE=$CURRENT_CMAKE_TOOLCHAIN_FILE \
  -DTBB_DIR=/path/to/oneTBB/install/lib/cmake/TBB \
  -DCMAKE_BUILD_TYPE=Release
cmake --build build --parallel
cmake --install build --prefix $PWD/install
  \end{lstlisting}
  \vspace{0.5em}
  \begin{itemize}
    \item Outputs: OpenVINO \texttt{.so}, plugins, \texttt{benchmark\_app} for target ABI.\@
    \item Keep install dir; you will push these binaries to the device.
  \end{itemize}
\end{frame}

\section{Deploy \& run}
\begin{frame}{Push to device}
  \begin{itemize}
    \item Enable developer mode + USB debugging; connect the device.
    \item \texttt{adb push} oneTBB \& OpenVINO \texttt{.so}, \texttt{libc++\_shared.so}, \texttt{benchmark\_app}, model into \texttt{/data/local/tmp}.
    \item Keep everything in one directory to simplify library lookup.
  \end{itemize}
\end{frame}

\begin{frame}[fragile]{Run inference}
  \begin{lstlisting}
adb shell "LD_LIBRARY_PATH=/data/local/tmp \
  /data/local/tmp/benchmark_app \
  -m /data/local/tmp/model.xml -hint latency"
  \end{lstlisting}
  \begin{itemize}
    \item \texttt{LD\_LIBRARY\_PATH} points to pushed libs; \texttt{-m} sets model; \texttt{-hint latency} optimizes for response time.
    \item Watch stdout for latency/FPS and any missing-library errors.
  \end{itemize}
\end{frame}

\section{Tuning}
\begin{frame}{Performance tips}
  \begin{itemize}
    \item Prefer FP16/INT8 models to reduce compute and bandwidth.
    \item Tune thread count (\texttt{-nthreads}) for each device.
    \item Balance load against thermals when picking thread count.
    \item Try AUTO/NNAPI plugins if the device supports them.
    \item Always ship \texttt{libc++\_shared.so} alongside your binaries.
  \end{itemize}
\end{frame}

\section{Wrap up}
\begin{frame}{What next}
  \begin{itemize}
    \item We built OpenVINO + oneTBB for Android and ran the demo binary.
    \item Next: package the \texttt{.so} into APK/AAB and call via JNI from app code.
    \item Check official OpenVINO Android guide for sample apps and plugin notes.
  \end{itemize}
\end{frame}

\end{document}
