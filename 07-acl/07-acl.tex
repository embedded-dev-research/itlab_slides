\documentclass{beamer}

% Theme choice
\usetheme{Madrid}

% Optional packages
\usepackage{graphicx} % For including images
\usepackage{amsmath}  % For math symbols and formulas
\usepackage{hyperref} % For hyperlinks

\title[ARM Compute Library introduction]{ARM Compute Library introduction}
\author{Nesterov Alexander, Obolenskiy Arseniy}
\institute{ITLab}

\date{\today}

% Redefine the footline to display both the short title and the org name
\setbeamertemplate{footline}{
  \leavevmode%
  \hbox{%
    \begin{beamercolorbox}[wd=.45\paperwidth,ht=2.5ex,dp=1ex,leftskip=1em,center]{author in head/foot}%
        \usebeamerfont{author in head/foot}\insertshortinstitute% Displays the university name
    \end{beamercolorbox}%
    \begin{beamercolorbox}[wd=.45\paperwidth,ht=2.5ex,dp=1ex,leftskip=1em,center]{author in head/foot}%
      \usebeamerfont{author in head/foot}\insertshorttitle% Displays the short title
    \end{beamercolorbox}%
    \begin{beamercolorbox}[wd=.1\paperwidth,ht=2.5ex,dp=1ex,rightskip=1em,center]{author in head/foot}%
      \usebeamerfont{author in head/foot}\insertframenumber{} / \inserttotalframenumber%
    \end{beamercolorbox}}%
  \vskip0pt%
}

\AtBeginSection[]{
  \begin{frame}
    \centering
    \Huge\insertsection%
  \end{frame}
}

\begin{document}

\begin{frame}
    \titlepage%
\end{frame}

\begin{frame}{Contents}
    \tableofcontents
\end{frame}

\section{ARM Compute Library}
\begin{frame}{ARM Compute Library}
  \begin{figure}[h]
    \includegraphics[width=0.8\textwidth]{images/acl.png}
  \end{figure}
  \footnotesize Source: \href{https://github.com/ARM-software/ComputeLibrary}{https://github.com/ARM-software/ComputeLibrary}
\end{frame}

\begin{frame}{Supported Architectures/Technologies}
  \begin{figure}[h]
    \includegraphics[width=1\textwidth]{images/arch.png}
  \end{figure}
  \footnotesize Source: \href{https://github.com/ARM-software/ComputeLibrary}{https://github.com/ARM-software/ComputeLibrary}
\end{frame}

\begin{frame}{Supported Systems}
  \begin{figure}[h]
    \includegraphics[width=1\textwidth]{images/os.png}
  \end{figure}
  \footnotesize Source: \href{https://github.com/ARM-software/ComputeLibrary}{https://github.com/ARM-software/ComputeLibrary}
\end{frame}

\section{Build ACL}
\begin{frame}{Build ACL}
  \begin{figure}[h]
    \includegraphics[width=1\textwidth]{images/build_macos.png}
  \end{figure}
  \begin{figure}[h]
    \includegraphics[width=1\textwidth]{images/exmpl.png}
  \end{figure}
  \footnotesize Source: \href{https://artificial-intelligence.sites.arm.com/computelibrary/latest/how_to_build.xhtml}{https://artificial-intelligence.sites.arm.com/computelibrary/latest/how\_to\_build.xhtml}
\end{frame}

\section{Return to examples}
\begin{frame}{Return to examples}
  \begin{figure}[h]
    \includegraphics[width=0.4\textwidth]{images/small_ir.png}
  \end{figure}
  \footnotesize Source: \href{https://docs.openvino.ai/}{docs.openvino.ai}
\end{frame}

\section{ACL operators}
\begin{frame}{ACL operators}
  \begin{figure}[h]
    \includegraphics[width=1\textwidth]{images/operators.png}
  \end{figure}
  \footnotesize Source: \href{https://artificial-intelligence.sites.arm.com/computelibrary/latest/operators_list.xhtml}{https://artificial-intelligence.sites.arm.com/computelibrary/latest/operators\_list.xhtml}
\end{frame}

\section{ACL activation operator}
\begin{frame}{ACL activation operator}
  \begin{figure}[h]
    \includegraphics[width=1\textwidth]{images/activation.png}
  \end{figure}
  \footnotesize Source: \href{https://artificial-intelligence.sites.arm.com/computelibrary/latest/operators_list.xhtml}{https://artificial-intelligence.sites.arm.com/computelibrary/latest/operators\_list.xhtml}
\end{frame}

\section{Validate activation operator}
\begin{frame}{Validate activation operator}
  \begin{figure}[h]
    \includegraphics[width=1\textwidth]{images/validate.png}
  \end{figure}
  \footnotesize Source: \href{https://artificial-intelligence.sites.arm.com/computelibrary/latest/classarm__compute_1_1_n_e_activation_layer.xhtml}{Description of activation operator}
\end{frame}

\section{TensorInfo for operators}
\begin{frame}{TensorInfo for operators}
  \begin{figure}[h]
    \includegraphics[width=1\textwidth]{images/tensorinfo.png}
  \end{figure}
  \footnotesize Source: \href{https://artificial-intelligence.sites.arm.com/computelibrary/latest/classarm__compute_1_1_tensor_info.xhtml}{Description of TensorInfo}
\end{frame}

\section{Configure activation operator}
\begin{frame}{Configure activation operator}
  \begin{figure}[h]
    \includegraphics[width=1\textwidth]{images/config.png}
  \end{figure}
  \footnotesize Source: \href{https://artificial-intelligence.sites.arm.com/computelibrary/latest/classarm__compute_1_1_n_e_activation_layer.xhtml}{Description of activation operator}
\end{frame}

\section{Tensor for operators}
\begin{frame}{Tensor for operators}
  \begin{figure}[h]
    \includegraphics[width=0.8\textwidth]{images/tensor.png}
  \end{figure}
  \footnotesize Source: \href{https://artificial-intelligence.sites.arm.com/computelibrary/latest/classarm__compute_1_1_tensor.xhtml}{Description of Tensor}
\end{frame}

\section{Run activation operator}
\begin{frame}{Run activation operator}
  \begin{figure}[h]
    \includegraphics[width=0.7\textwidth]{images/run.png}
  \end{figure}
  \footnotesize Source: \href{https://artificial-intelligence.sites.arm.com/computelibrary/latest/classarm__compute_1_1_n_e_activation_layer.xhtml}{Description of activation operator}
\end{frame}

\section{Get ONNX model}
\begin{frame}{Get ONNX model}
  \begin{figure}[h]
    \includegraphics[width=1\textwidth]{images/ultralytics.png}
  \end{figure}
  \footnotesize Source: \href{https://docs.ultralytics.com/integrations/onnx/}{https://docs.ultralytics.com/integrations/onnx/}
\end{frame}

\end{document}
